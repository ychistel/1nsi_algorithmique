\documentclass[11pt,a4paper]{article}

\usepackage{style2017}
\usepackage{hyperref}
\newcommand{\esp}{\hspace{0.5cm}}
\newcommand{\espp}{\hspace{1cm}}
\newcommand{\esppp}{\hspace{1.5cm}}

\hypersetup{
    colorlinks =false,
    linkcolor=blue,
   linkbordercolor = 1 0 0
}
\newcounter{numexo}
\setcellgapes{1pt}


\begin{document}

\begin{NSI}
{Exercice}{Recherche par dichotomie}
\end{NSI}


%\addtocounter{numexo}{1}
%\subsection*{\Large Exercice \thenumexo}
%Max et Lilly joue à devine nombre. Max pense à un nombre entier compris entre 1 et 1000 et Lilly doit le trouver. À chaque nombre proposé par Lilly, Max lui dit s'il est plus petit ou plus grand. 
%
%Si Lilly applique la recherche par dichotomie, en combien d'essais au maximum peut-elle trouver le nombre de Max ?


\addtocounter{numexo}{1}
\subsection*{\Large Exercice \thenumexo}
Soit T un tableau trié tel que: $T=[8, 8, 17, 21, 23, 27, 28, 45, 57, 71, 77, 84, 88, 95, 97]$.
\begin{enumerate}
\item On veut vérifier la présence du nombre $23$ dans le tableau.
\begin{enumerate}
\item Retracer toutes les étapes de l'algorithme de recherche par dichotomie.
\item Combien d'itérations ont été nécessaires?
\end{enumerate}
\item Faire de même avec le nombre $75$.
\end{enumerate}


\addtocounter{numexo}{1}
\subsection*{\Large Exercice \thenumexo}

\begin{enumerate}
\item Créer un tableau $T$ de valeurs choisies aléatoirement entre 1 et 1000.
\item Trier ce tableau avec une des fonctions de tri de python.
\item Écrire l'algorithme de recherche par dichotomie en langage python.
\item Tester votre programme avec la recherche de différentes valeurs du tableau et des valeurs qui ne sont pas dans le tableau.

\item Modifier votre fonction pour renvoyer le nombre d'itérations nécessaires à la recherche d'une valeur.
\item Proposer une version récursive de l'algorithme de recherche par dichotomie.
\end{enumerate}


\end{document}


